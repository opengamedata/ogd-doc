\documentclass{beamer}
% Based on template by Till Tantau <tantau@users.sourceforge.net>.

% \usepackage[english]{babel}
\usepackage[latin1]{inputenc}
% or whatever
\usepackage[T1]{fontenc}
% Or whatever. Note that the encoding and the font should match. If T1
% does not look nice, try deleting the line with the fontenc.

\mode<presentation>
{
  \usetheme{frankfurt}
  % or ...

  \setbeamercovered{transparent}
  % or whatever (possibly just delete it)
}

\pgfdeclareimage[height=0.5cm]{ogd-logo}{../../../assets/OpenGameData-logo.png}
\logo{\pgfuseimage{ogd-logo}}

\title[Intro to OGD] % (optional, use only with long paper titles)
{Lecture 1: Introduction to Open Game Data}
\subtitle
{Who it is, who it's for, who should view what units} % (optional)

\author[Swanson] % (optional, use only with lots of authors)
{Luke Swanson}
% - Use the \inst{?} command only if the authors have different
%   affiliation.

\institute[University of Wisconsin-Madison] % (optional, but mostly needed)
{
  Field Day Lab\\
  University of Wisconsin-Madison
}
% - Use the \inst command only if there are several affiliations.
% - Keep it simple, no one is interested in your street address.

\date[OGD Docs] % (optional)
{OpenGameData Documentation}

\subject{Talks}
% This is only inserted into the PDF information catalog. Can be left out. 

% Delete this, if you do not want the table of contents to pop up at
% the beginning of each subsection:
\AtBeginSubsection[]
{
  \begin{frame}<beamer>{Outline}
    \tableofcontents[currentsection,currentsubsection]
  \end{frame}
}


% If you wish to uncover everything in a step-wise fashion, uncomment
% the following command: 
\beamerdefaultoverlayspecification{<+->}


\begin{document}

\begin{frame}
  \titlepage
\end{frame}

\begin{frame}{Overview}
  \tableofcontents
  % You might wish to add the option [pausesections]
\end{frame}


% Since this a solution template for a generic talk, very little can
% be said about how it should be structured. However, the talk length
% of between 15min and 45min and the theme suggest that you stick to
% the following rules:  

% - Exactly two or three sections (other than the summary).
% - At *most* three subsections per section.
% - Talk about 30s to 2min per frame. So there should be between about
%   15 and 30 frames, all told.

\section[Questions]{5 W's and an H}

\begin{frame}{What is Open Game Data?}
  \begin{itemize}
  \item
    // Discuss Open Game Data community vs. OpenGameData software
    \pause
  \end{itemize}
\end{frame}

\begin{frame}{Who is Open Game Data?}
  \begin{itemize}
  \item
    // Introduce major members of OGD community
    \pause
  \end{itemize}
\end{frame}

\begin{frame}{Where is Open Game Data?}
  \begin{itemize}
    \item // Re-display list from Who? slide
    \pause
    \item // Note what Universities and groups our members are associated with
    \pause
  \end{itemize}
\end{frame}

\begin{frame}{Why is Open Game Data?}
  \begin{itemize}
  \item
    // Summarize the mission of OGD
    \pause
  \end{itemize}
\end{frame}

\begin{frame}{When is Open Game Data?}
  \begin{itemize}
  \item
    // Mention briefly the existence of biweekly office hours
    \pause
  \end{itemize}
\end{frame}

\begin{frame}{How is Open Game Data?}
  \begin{itemize}
  \item
    // Mention where we get funding from, what makes it all work.
    \pause
  \end{itemize}
\end{frame}

\section[Documentation]{Navigating the Documentation}

\begin{frame}{Documentation Organization}
  \begin{itemize}
  \item
    // Explain breakdown of units
    \pause
  \end{itemize}
\end{frame}

\begin{frame}{Stakeholders}{Who are you?}
  \begin{itemize}
  \item
    // Describe a few potential users of the documentation,
    and which units they should view.
    \pause
  \end{itemize}
\end{frame}

\begin{frame}{Additional Resources}
  \begin{itemize}
  \item
    // Mention where to find our readthedocs, which has similar structure and more details.
    \pause
  \end{itemize}
\end{frame}

\section[OGD Pipeline]{The OpenGameData Pipeline}

\begin{frame}{Pipeline Overview}
  \begin{itemize}
  \item
    // Give a basic outline of the pipeline, from game to feature extractors
    \pause
  \end{itemize}
\end{frame}

\begin{frame}{Pipeline Terminology}{Events, Detectors, and Features}
  \begin{itemize}
  \item
    // Discuss how events form sessions, sessions to players, players to populations
    \pause
  \end{itemize}
\end{frame}

\begin{frame}{Future Extensions}
  \begin{itemize}
  \item
    // Mention that in the future, we'll work on supporting training and testing of models
    \pause
  \end{itemize}
\end{frame}

\section[Getting Started]{Getting Started with OGD}

\begin{frame}{The Open Game Data Website}
  \begin{itemize}
  \item
    // Brief tour of the OGD website, and getting datasets
    \pause
  \end{itemize}
\end{frame}

\begin{frame}{Research Samples \& Templates}
  \begin{itemize}
  \item
    // Show where samples and templates are, mention we'll discuss more in another video
    \pause
  \end{itemize}
\end{frame}

\section*{Summary}

\begin{frame}{Summary}
  % Keep the summary *very short*.
  \begin{itemize}
  \item
    The \alert{first main message} of your talk in one or two lines.
  \item
    The \alert{second main message} of your talk in one or two lines.
  \item
    Perhaps a \alert{third message}, but not more than that.
  \end{itemize}
\end{frame}


\end{document}


