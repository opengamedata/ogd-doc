\documentclass{beamer}
% Based on template by Till Tantau <tantau@users.sourceforge.net>.

% \usepackage[english]{babel}
\usepackage[latin1]{inputenc}
% or whatever
\usepackage[T1]{fontenc}
% Or whatever. Note that the encoding and the font should match. If T1
% does not look nice, try deleting the line with the fontenc.
\usepackage{graphicx}

\mode<presentation>
{
  \usetheme{frankfurt}
  \usepackage{../../theme/beamercolorthemeogd}
  % or ...

  \setbeamercovered{transparent}
  % or whatever (possibly just delete it)
}

\title[Intro to Events] % (optional, use only with long paper titles)
{Lecture 2: Introduction to Event Logging in OpenGameData}

\author[Swanson] % (optional, use only with lots of authors)
{Luke Swanson}
\institute[University of Wisconsin-Madison] % (optional, but mostly needed)
{
  Field Day Lab\\
  University of Wisconsin-Madison
}

\date[OGD Docs] % (optional)
{OpenGameData Documentation \\ Unit 1: Event Logging}

\subject{Talks}
% This is only inserted into the PDF information catalog. Can be left out. 

% Delete this, if you do not want the table of contents to pop up at
% the beginning of each subsection:
\AtBeginSubsection[]
{
  \begin{frame}<beamer>{Outline}
    \tableofcontents[currentsection,currentsubsection]
  \end{frame}
}

% If you wish to uncover everything in a step-wise fashion, uncomment
% the following command: 
\beamerdefaultoverlayspecification{<+->}


\begin{document}

\begin{frame}
  \titlepage
\end{frame}

\begin{frame}{Overview}
  \tableofcontents
  % You might wish to add the option [pausesections]
\end{frame}


% Since this a solution template for a generic talk, very little can
% be said about how it should be structured. However, the talk length
% of between 15min and 45min and the theme suggest that you stick to
% the following rules:  

% - Exactly two or three sections (other than the summary).
% - At *most* three subsections per section.
% - Talk about 30s to 2min per frame. So there should be between about
%   15 and 30 frames, all told.

\section[Event Schema]{The OpenGameData Event Schema}

\begin{frame}{What is the OGD Event Schema?}
  \begin{itemize}
  \item Events vary greatly across games
  \item One standard allows consistent interface across games
  \end{itemize}
\end{frame}

\begin{frame}{OGD Event Schema Categories}
  \begin{itemize}
  \item IDs
  \item Versioning
  \item Timing
  \item Data (the good stuff!)
  \end{itemize}
\end{frame}

\subsection[IDs]{Common Elements}

\begin{frame}{What Identifiers Do We Use?}
  \begin{itemize}
  \item Player
  \item Session
  \end{itemize}
\end{frame}

\begin{frame}{What Versioning Do We Use?}
  \begin{itemize}
  \item Player
  \item Session
  \end{itemize}
\end{frame}

\section[Examples]{Example Games}

\begin{frame}{Simple Example Game}
  \begin{itemize}
  \item Something about an example of a really simple game
  \item So we can imagine doing a thing with it
  \end{itemize}
\end{frame}

\section*{Summary}

\begin{frame}{Summary}
  % Keep the summary *very short*.
  \begin{itemize}
  \item
    The \alert{first main message} of your talk in one or two lines.
  \item
    The \alert{second main message} of your talk in one or two lines.
  \item
    Perhaps a \alert{third message}, but not more than that.
  \end{itemize}
\end{frame}


\end{document}


